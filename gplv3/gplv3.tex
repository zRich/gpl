% !TeX program  = XeLaTeX
\documentclass[11pt]{article}
\usepackage[UTF8]{ctex}
% \usepackage{ctex}
% 中文默认字体:思源宋体,粗体为思源宋体粗体,斜体为方正楷体_GBK
\setCJKmainfont{Source Han Serif SC}[BoldFont=Source Han Serif SC SemiBold]
% % 中文无衬线字体:思源黑体,粗体为思源黑体粗体
\setCJKsansfont{Source Han Sans SC}[BoldFont={Source Han Sans SC Bold}]
% % 中文等宽字体:Courier
\setCJKmonofont{Courier}
\title{GNU通用公开许可证}
\date{第3版 \space 2007年6月29日}

\begin{document}
\maketitle

\begin{center}
{\parindent 0in

版权所有 \copyright\  2007年 自由软件基金会 \texttt{https://fsf.org/}

\bigskip
任何人都可以复制和发布本许可证完整副本,但不允许修改。}

\end{center}
\begin{center}
  {\bf\large 译者声明}
\end{center}

This is an unofficial translation of the GNU General Public License into Chinese. It was not published by the Free Software Foundation, and does not legally state the distribution terms for software that uses the GNU GPL--only the original English text of the GNU GPL does that. However, we hope that this translation will help Chinese speakers understand the GNU GPL better.

本译文是GNU通用公共许可协议的一份非官方中文翻译,并非自由软件基金会所发表,不适用于使用GNU通用公共许可协议发布的软件的法律声明——只有GNU通用公共许可协议英文原版才具有法律效力。不过我们希望本翻译能够帮助中文读者更好地理解GNU通用公共许可协议。

仅在遵循 <https://www.gnu.org/licenses/translations.html> 中的条款时,你才可以经过修改地或者不经过修改地发布本译文。

\pagebreak

\renewcommand{\abstractname}{引言}
\begin{abstract}
The GNU General Public License is a free, copyleft license for
software and other kinds of works.
GNU通用公共许可协议是一份面向软件及其他类型作品的、自由的版权公共协议。

The licenses for most software and other practical works are designed
to take away your freedom to share and change the works.  By contrast,
the GNU General Public License is intended to guarantee your freedom to
share and change all versions of a program--to make sure it remains free
software for all its users.  We, the Free Software Foundation, use the
GNU General Public License for most of our software; it applies also to
any other work released this way by its authors.  You can apply it to
your programs, too.

就多数软件而言,许可协议被设计用于剥夺你分享和修改软件的自由。相反,GNU通用公共许可协议力图保障你分享和修改某程序全部版本的权利——确保自由软件对其用户来说是自由的。我们自由软件基金会将GNU通用公共许可协议用于我们的大多数软件,并为一些其他作品的作者效仿。你也可以将本协议用于你的程序。

When we speak of free software, we are referring to freedom, not
price.  Our General Public Licenses are designed to make sure that you
have the freedom to distribute copies of free software (and charge for
them if you wish), that you receive source code or can get it if you
want it, that you can change the software or use pieces of it in new
free programs, and that you know you can do these things.

所谓自由软件,强调自由,而非免费。本GNU通用公共许可协议设计用于确保你享有分发自由软件的自由(你可以为此服务收费),确保你可以在需要的时候获得这些软件的源码,确保你可以修改这些软件或者在新的自由软件中复用其中某些片段,并且确保你在这方面享有知情权。

To protect your rights, we need to prevent others from denying you
these rights or asking you to surrender the rights.  Therefore, you have
certain responsibilities if you distribute copies of the software, or if
you modify it: responsibilities to respect the freedom of others.

为了保护你的权利,我们设置了一些限制以防止其他人否定你的权利或者要求你放弃你的权利。这些限制在你分发或者修改这些软件时会成为你的责任。

For example, if you distribute copies of such a program, whether
gratis or for a fee, you must pass on to the recipients the same
freedoms that you received.  You must make sure that they, too, receive
or can get the source code.  And you must show them these terms so they
know their rights.

例如,你分发这类软件的副本,无论是收费或者免费,你必须授予接收者你拥有的所有权利。你必须保证他们也能收到或者能够获得源代码。并且你也要确保他们也知道他们的权利。

Developers that use the GNU GPL protect your rights with two steps:
(1) assert copyright on the software, and (2) offer you this License
giving you legal permission to copy, distribute and/or modify it.

采用GNU通用公共许可协议的开发者通过两步保障你的权利:(1)申明软件的版权;(2)通过本协议授予你合法地复制、分发和修改该软件的权利。

For the developers' and authors' protection, the GPL clearly explains
that there is no warranty for this free software.  For both users' and
authors' sake, the GPL requires that modified versions be marked as
changed, so that their problems will not be attributed erroneously to
authors of previous versions.

为了保护作者和开发者,GNU通用公共许可协议声明:自由软件并没有品质担保。为用户和作者双方着想,GNU通用公共许可协议要求修改版必须有标记,以免其问题被错误地归到先前版本的作者身上。

Some devices are designed to deny users access to install or run
modified versions of the software inside them, although the manufacturer
can do so.  This is fundamentally incompatible with the aim of
protecting users' freedom to change the software.  The systematic
pattern of such abuse occurs in the area of products for individuals to
use, which is precisely where it is most unacceptable.  Therefore, we
have designed this version of the GPL to prohibit the practice for those
products.  If such problems arise substantially in other domains, we
stand ready to extend this provision to those domains in future versions
of the GPL, as needed to protect the freedom of users.

某些设备设计成拒绝用户安装、运行修改过的软件,但厂商不受此限制。这和我们保护用户享有修改软件的自由的宗旨存在根本性矛盾。
该滥用协议的模式出现于个人用品领域,这恰恰是最不可接受的。因此,我们设计了这版GNU通用公共许可协议来禁止这种做法。
如果此类问题在其他领域出现,我们时刻准备着在将来的版本中把规定扩展到相应领域,以保护用户的自由。

Finally, every program is threatened constantly by software patents.
States should not allow patents to restrict development and use of
software on general-purpose computers, but in those that do, we wish to
avoid the special danger that patents applied to a free program could
make it effectively proprietary.  To prevent this, the GPL assures that
patents cannot be used to render the program non-free.

最后,每个程序都持续受到软件专利的威胁。政府不应该允许专利限制通用计算机软件的开发和应用,在做不到这点时,我们希望避免因专利应用而使自由软件私有化的危险。
就此,GNU通用公共许可协议保证专利不能使程序非自由化。

The precise terms and conditions for copying, distribution and
modification follow.
下文是关于复制、分发和修改的严谨描述和实施条件。

\end{abstract}

\begin{center}
{\Large \sc 条款和条件}
\end{center}


\begin{enumerate}

\addtocounter{enumi}{-1}

\item 定义

``This License'' refers to version 3 of the GNU General Public License.

“本许可”指GNU通用公共许可协议第3版。

``Copyright'' also means copyright-like laws that apply to other kinds of
works, such as semiconductor masks.

“版权”也指适用于诸如半导体掩模的其他类型作品的类似法律。

``The Program'' refers to any copyrightable work licensed under this
License.  Each licensee is addressed as ``you''.  ``Licensees'' and
``recipients'' may be individuals or organizations.

“本程序”指任何在本协议保护下的有版权的作品。每个许可获得者称作“你”。“许可获得者”和“接收者”可以是个人或组织。

To ``modify'' a work means to copy from or adapt all or part of the work
in a fashion requiring copyright permission, other than the making of an
exact copy.  The resulting work is called a ``modified version'' of the
earlier work or a work ``based on'' the earlier work.

“修改”一个作品指需要版权许可的复制及对作品全面的或部分的改编行为,有别于制作副本。所产生的作品称作前作的“修改版”,或“基于”前作的作品。

A ``covered work'' means either the unmodified Program or a work based
on the Program.
“受保护作品”指程序或其派生作品。

To ``propagate'' a work means to do anything with it that, without
permission, would make you directly or secondarily liable for
infringement under applicable copyright law, except executing it on a
computer or modifying a private copy.  Propagation includes copying,
distribution (with or without modification), making available to the
public, and in some countries other activities as well.
“传播”作品指那些未经许可就会在适用版权法律下构成直接或间接侵权的行为,不包括在计算机上运行和私下的修改。
传播包括复制、分发(无论修改与否)、向公众公开,以及在某些国家的其他行为。
To ``convey'' a work means any kind of propagation that enables other
parties to make or receive copies.  Mere interaction with a user through
a computer network, with no transfer of a copy, is not conveying.
“转发”作品指让他方能够制作或者接收副本的行为。仅仅通过计算机网络和用户交互,没有传输副本,则不算转发。

An interactive user interface displays ``Appropriate Legal Notices''
to the extent that it includes a convenient and prominently visible
feature that (1) displays an appropriate copyright notice, and (2)
tells the user that there is no warranty for the work (except to the
extent that warranties are provided), that licensees may convey the
work under this License, and how to view a copy of this License.  If
the interface presents a list of user commands or options, such as a
menu, a prominent item in the list meets this criterion.

一个显示“适当的法律声明”的交互式用户界面应包括一个便捷而醒目的可视化特性:(1)显示适当的版权声明;(2)告知用户没有品质担保(提供了品质担保的情况除外),许可获得者可以在本协议约束下转发该作品,及查看本协议副本的途径。如果该界面提供一个命令列表,如菜单,其表项应符合上述规范。


\item 源代码

The ``source code'' for a work means the preferred form of the work
for making modifications to it.  ``Object code'' means any non-source
form of a work.

作品的“源代码”指其可修改的首选形式,目标代码指所有其他形式。

A ``Standard Interface'' means an interface that either is an official
standard defined by a recognized standards body, or, in the case of
interfaces specified for a particular programming language, one that
is widely used among developers working in that language.

“标准接口”指标准化组织定义的官方标准中的接口,或针为某种编程语言设定的接口中为开发者广泛使用的接口。

The ``System Libraries'' of an executable work include anything, other
than the work as a whole, that (a) is included in the normal form of
packaging a Major Component, but which is not part of that Major
Component, and (b) serves only to enable use of the work with that
Major Component, or to implement a Standard Interface for which an
implementation is available to the public in source code form.  A
``Major Component'', in this context, means a major essential component
(kernel, window system, and so on) of the specific operating system
(if any) on which the executable work runs, or a compiler used to
produce the work, or an object code interpreter used to run it.

可执行作品中的“系统库”不是指整个程序,而是涵盖此等内容:(a)以通常形式和主部件打包到一起却并非后者一部分,
且(b)仅为和主部件一起使作品可用或实现某些已有公开实现源码的接口。“主部件”在这里指可执行作品运行依赖的操作系统(如果存在)的必要部件(内核、窗口系统等),或者生成该作品的编译器,或运行所需的目标代码解释器。

The ``Corresponding Source'' for a work in object code form means all
the source code needed to generate, install, and (for an executable
work) run the object code and to modify the work, including scripts to
control those activities.  However, it does not include the work's
System Libraries, or general-purpose tools or generally available free
programs which are used unmodified in performing those activities but
which are not part of the work.  For example, Corresponding Source
includes interface definition files associated with source files for
the work, and the source code for shared libraries and dynamically
linked subprograms that the work is specifically designed to require,
such as by intimate data communication or control flow between those
subprograms and other parts of the work.

目标代码形式作品的“相应的源码”指所有修改作品及生成、安装、运行(对可执行作品而言)目标代码所需的源码,或者修改作品的源代码,包括控制上述行为的脚本。
可是,其中不包括系统库、通用工具、不需要修改就可以直接用于支持上述行为但不是该作品一部分的、通常可得的自由软件。
例如,相应的源码包含与作品源文件相关的接口定义,以及共享库和作品专门依赖的动态链接子程序的源码。
这里的依赖体现为密切的数据交换或者该子程序和作品其他部分的控制流切换。

The Corresponding Source need not include anything that users
can regenerate automatically from other parts of the Corresponding
Source.

相应的源码不必包含那些用户可以通过源码其他部分自动生成的内容。


The Corresponding Source for a work in source code form is that
same work.

源码形式作品的相应源码即其本身。

\item Basic Permissions.
\item 基本许可

All rights granted under this License are granted for the term of
copyright on the Program, and are irrevocable provided the stated
conditions are met.  This License explicitly affirms your unlimited
permission to run the unmodified Program.  The output from running a
covered work is covered by this License only if the output, given its
content, constitutes a covered work.  This License acknowledges your
rights of fair use or other equivalent, as provided by copyright law.

本协议的所有授权都是对本程序的版权而言的,并且当所述条件都满足时不可撤销。本协议明确授权你不受限制地运行本程序的未修改版本。
运行受保护作品的输出结果,仅当其内容构成一个受保护作品时,才受本协议所约束。如版权法所授权一样,本协议承认你正当使用或与之等价的权利。


You may make, run and propagate covered works that you do not
convey, without conditions so long as your license otherwise remains
in force.  You may convey covered works to others for the sole purpose
of having them make modifications exclusively for you, or provide you
with facilities for running those works, provided that you comply with
the terms of this License in conveying all material for which you do
not control copyright.  Those thus making or running the covered works
for you must do so exclusively on your behalf, under your direction
and control, on terms that prohibit them from making any copies of
your copyrighted material outside their relationship with you.

只要你获得的许可仍有效,你就可以制作、运行和传播受保护作品,不得传递。
在你遵守本协议中关于转发你拥有版权的材料的条款时,你可以向他人传递受保护的作品,以让对方单独为你定制修改,或者向你提供运行这些作品的工具。
那些为你制作或运行这些受保护作品的人,必须在你的指引和控制下,仅代表你工作,即禁止他们在双方关系之外制作任何你提供的受版权保护材料的副本。


Conveying under any other circumstances is permitted solely under
the conditions stated below.  Sublicensing is not allowed; section 10
makes it unnecessary.

仅当满足后文所述条件时,其他各种情况下的传递才是被允许的。不允许再授权,而第10条的存在使再授权变得没有必要。

\item Protecting Users' Legal Rights From Anti-Circumvention Law.
\item 保护用户的合法权益免受反破解法限制

No covered work shall be deemed part of an effective technological
measure under any applicable law fulfilling obligations under article
11 of the WIPO copyright treaty adopted on 20 December 1996, or
similar laws prohibiting or restricting circumvention of such
measures.

为了履行1996年12月20日通过的WIPO版权条约第11章规定的义务,
法律规定了禁止或规避措施的条款,所有受保护作品不应该被视为规避这些法律条款的技术手段。

When you convey a covered work, you waive any legal power to forbid
circumvention of technological measures to the extent such circumvention
is effected by exercising rights under this License with respect to
the covered work, and you disclaim any intention to limit operation or
modification of the work as a means of enforcing, against the work's
users, your or third parties' legal rights to forbid circumvention of
technological measures.

如果你传递一个受保护作品,即表明你放弃禁止技术规避措施的法律权利,行使本许可协议所授予权利可以实现规避,同时,
你也放弃禁止技术规避措施相关的法律赋予你或者第三方限制运行或者修改本作品的权利。

\item Conveying Verbatim Copies.
\item 传递原始副本

You may convey verbatim copies of the Program's source code as you
receive it, in any medium, provided that you conspicuously and
appropriately publish on each copy an appropriate copyright notice;
keep intact all notices stating that this License and any
non-permissive terms added in accord with section 7 apply to the code;
keep intact all notices of the absence of any warranty; and give all
recipients a copy of this License along with the Program.

你可以通过任何媒介发布你接收到的本程序的完整源码副本,但必须做到:为每一个副本醒目而恰当地发布版权声明;
完整地保留关于本许可及按第7条加入的非许可性条款;完整地保留所有免责声明;给接收者附上一份本协议的副本。


You may charge any price or no price for each copy that you convey,
and you may offer support or warranty protection for a fee.

你可以免费或收任何费用传递,也可以选择提供技术支持或品质担保以收取费用。

\item Conveying Modified Source Versions.
\item 传递经过修改的源代码

You may convey a work based on the Program, or the modifications to
produce it from the Program, in the form of source code under the
terms of section 4, provided that you also meet all of these conditions:
你可以以第4条规定的源代码形式传递基于本程序的作品或修改的内容,但必须满足以下要求:
  \begin{enumerate}
  \item The work must carry prominent notices stating that you modified
  it, and giving a relevant date.

  \item 该作品必须带有醒目的修改声明及相应的日期。

  \item The work must carry prominent notices stating that it is
  released under this License and any conditions added under section
  7.  This requirement modifies the requirement in section 4 to
  ``keep intact all notices''.

  \item 该作品必须带有醒目的声明,指明其在本协议及任何符合第7条的附加条款下发布。这个要求修正了第4条关于“完整保留所有声明”的内容。

  \item You must license the entire work, as a whole, under this
  License to anyone who comes into possession of a copy.  This
  License will therefore apply, along with any applicable section 7
  additional terms, to the whole of the work, and all its parts,
  regardless of how they are packaged.  This License gives no
  permission to license the work in any other way, but it does not
  invalidate such permission if you have separately received it.

  \item 你必须按照本许可将该作品整体授权给任何得到副本的人。本协议及符合第7条的附加条款适用于整个作品,即其每一部分,不管它是如何组建的。
  本许可不允许以其他形式授权本作品,但不会使用已经单独的其他许可无效。

  \item If the work has interactive user interfaces, each must display
  Appropriate Legal Notices; however, if the Program has interactive
  interfaces that do not display Appropriate Legal Notices, your
  work need not make them do so.

  \item 如果该作品有交互式用户界面,则其必须显示适当的法律声明。然而,当程序有交互式用户界面却不显示适当的法律声明时,你的作品也无需使其显示。

\end{enumerate}
A compilation of a covered work with other separate and independent
works, which are not by their nature extensions of the covered work,
and which are not combined with it such as to form a larger program,
in or on a volume of a storage or distribution medium, is called an
``aggregate'' if the compilation and its resulting copyright are not
used to limit the access or legal rights of the compilation's users
beyond what the individual works permit.  Inclusion of a covered work
in an aggregate does not cause this License to apply to the other
parts of the aggregate.

一个在存储或分发介质上的受保护作品和其他分离的单体作品的联合作品,在既不是该受保护作品的自然扩展,
也不以构筑更大的程序为目的,并且自身及其产生的版权并非用于限制单体作品给予联合作品用户的访问及其他合法权利时,称为“聚合体”。
在聚合作品中包含受保护作品并不会使本协议影响聚合作品的其他部分。

一个受保护作品与其他分离且单独的作品组成一个组合,其中的单独既不是受保护作品的自然延伸,也不是为了与受保护作品组成更大程序而与被保护作品存储或者分发介质上,
并且这种组合和组合后的版权不会限制单独作品的授权,则这个组合称为“组合作品”。

\item Conveying Non-Source Forms.
\item 以非源代码形式传递

You may convey a covered work in object code form under the terms
of sections 4 and 5, provided that you also convey the
machine-readable Corresponding Source under the terms of this License,
in one of these ways:
你可以以第4条和第5条所述那样以目标代码形式传递受保护作品,同时在本许可规范下以如下方式之一传递机器可读的对应源代码:

  \begin{enumerate}
  \item Convey the object code in, or embodied in, a physical product
  (including a physical distribution medium), accompanied by the
  Corresponding Source fixed on a durable physical medium
  customarily used for software interchange.
  \item 通过物理产品(包括物理分发媒介)传递或者嵌入目标代码时,通过常用于软件交换的耐用型物理媒介随同传递相应的源代码。

  \item Convey the object code in, or embodied in, a physical product
  (including a physical distribution medium), accompanied by a
  written offer, valid for at least three years and valid for as
  long as you offer spare parts or customer support for that product
  model, to give anyone who possesses the object code either (1) a
  copy of the Corresponding Source for all the software in the
  product that is covered by this License, on a durable physical
  medium customarily used for software interchange, for a price no
  more than your reasonable cost of physically performing this
  conveying of source, or (2) access to copy the
  Corresponding Source from a network server at no charge.

  \item 通过物理产品(包括物理分发媒介)时,附随具有至少3年有效期的书面承诺,并且有效期涵盖提供的备件或客户支持,以授予任何目标代码的持有者:
  (1)获得全部受保护产品全部软件的相应源代码的副本,副本通过常用于软件交换的耐用型物理媒介提供,且收费不超过其合理的传递成本;
  或者(2)通过网络免费获得相应源代码的途径。

  \item Convey individual copies of the object code with a copy of the
  written offer to provide the Corresponding Source.  This
  alternative is allowed only occasionally and noncommercially, and
  only if you received the object code with such an offer, in accord
  with subsection 6b.
  
  \item 单独传递目标代码的副本时,伴以提供源代码的书面承诺。本选项仅在偶尔并且非商业情况下可以使用,同时你收到也是第6条b项所述的目标代码的情况下可用。
  
  \item Convey the object code by offering access from a designated
  place (gratis or for a charge), and offer equivalent access to the
  Corresponding Source in the same way through the same place at no
  further charge.  You need not require recipients to copy the
  Corresponding Source along with the object code.  If the place to
  copy the object code is a network server, the Corresponding Source
  may be on a different server (operated by you or a third party)
  that supports equivalent copying facilities, provided you maintain
  clear directions next to the object code saying where to find the
  Corresponding Source.  Regardless of what server hosts the
  Corresponding Source, you remain obligated to ensure that it is
  available for as long as needed to satisfy these requirements.

  \item 通过在指定地点提供目标代码获取服务(无论是否收费)的形式传递目标代码时,在同一地点以同样的方式提供相同的源码获取服务,
  并不得额外收费。你不必要求接收者在复制目标代码的同时复制源码。如果提供获取目标代码的地址为网络服务器,相应的源代码可以提供在另一个支持相同复制功能的服务器
  上(由你或者第三方运营),不过你要在目标代码处指出相应源码的确切路径。不管你用什么源代码服务器,你有义务要确保持续可用以满足这些要求。

  \item Convey the object code using peer-to-peer transmission, provided
  you inform other peers where the object code and Corresponding
  Source of the work are being offered to the general public at no
  charge under subsection 6d.
  \item 通过点对点传输传递目标代码时,告知其他节点目标代码和源代码在何处,并以第6条d项形式向大众免费提供。

  \end{enumerate}

A separable portion of the object code, whose source code is excluded
from the Corresponding Source as a System Library, need not be
included in conveying the object code work.

目标代码的可分离部分,其源代码被排除在外来自相应的源作为系统库,不需要包括传送目标代码的工作。

A ``User Product'' is either (1) a ``consumer product'', which means any
tangible personal property which is normally used for personal, family,
or household purposes, or (2) anything designed or sold for incorporation
into a dwelling.  In determining whether a product is a consumer product,
doubtful cases shall be resolved in favor of coverage.  For a particular
product received by a particular user, ``normally used'' refers to a
typical or common use of that class of product, regardless of the status
of the particular user or of the way in which the particular user
actually uses, or expects or is expected to use, the product.  A product
is a consumer product regardless of whether the product has substantial
commercial, industrial or non-consumer uses, unless such uses represent
the only significant mode of use of the product.

“用户产品”指(1)“消费品”,即个人、家庭或日常用途的个人的有形财产;或者(2)面向家庭设计或销售的物品。在判断一款产品是否消费品时,
对于有争议的案例,应扩大范围尽量涵盖。就特定用户接收到特定产品而言,“正常使用”指对此类产品的典型的或一般使用,与该用户的身份,
该用户对该产品的实际用法,以及该产品的预期用法无关。无论产品是否实质上具有商业上的,工业上的,及非面向消费者的用法,它都视为消费品,
除非以上用法代表了它唯一的重要使用模式。


``Installation Information'' for a User Product means any methods,
procedures, authorization keys, or other information required to install
and execute modified versions of a covered work in that User Product from
a modified version of its Corresponding Source.  The information must
suffice to ensure that the continued functioning of the modified object
code is in no case prevented or interfered with solely because
modification has been made.

用户产品的“安装信息”,指基于修改过的源代码用来安装运行该产品中的受保护作品的修改版所需的方法、
流程、授权秘钥及其他信息。这些信息必须足以保证修改过的目标代码不会仅仅因为被修改过而不能继续工作。

If you convey an object code work under this section in, or with, or
specifically for use in, a User Product, and the conveying occurs as
part of a transaction in which the right of possession and use of the
User Product is transferred to the recipient in perpetuity or for a
fixed term (regardless of how the transaction is characterized), the
Corresponding Source conveyed under this section must be accompanied
by the Installation Information.  But this requirement does not apply
if neither you nor any third party retains the ability to install
modified object code on the User Product (for example, the work has
been installed in ROM).

如果你根据本条规定,传递一个目标代码作品到用户产品中或与用户产品,或专门用于特定用户产品的目的,并且传送行为成为交易的一部分,
使得用户产品永久或在特定期限内暂时转移给接收者(无论交易的特征如何),必须通过安装信息附上根据本条传递的相应来源。 
但如果你和第三方都没有在用户产品中保留安装修改目标代码的能力时,可以不遵守该要求(例如,作品有已安装在 ROM 中)。

The requirement to provide Installation Information does not include a
requirement to continue to provide support service, warranty, or updates
for a work that has been modified or installed by the recipient, or for
the User Product in which it has been modified or installed.  Access to a
network may be denied when the modification itself materially and
adversely affects the operation of the network or violates the rules and
protocols for communication across the network.

这项要求不成立。要求提供安装信息并不要求为修改或安装的作品,以及其载体产品继续提供技术支持、品质担保和升级。
当修改本身对网络运行有实质上的负面影响,或违背了网络通信协议和规则时,可以拒绝其联网。

Corresponding Source conveyed, and Installation Information provided,
in accord with this section must be in a format that is publicly
documented (and with an implementation available to the public in
source code form), and must require no special password or key for
unpacking, reading or copying.

根据本条规定发布的源代码及安装信息,必须以公共的文档格式(并且以源代码形式实现对公众可用)存在,同时不得对解压、阅读和复制设置任何密码或秘钥。

\item Additional Terms.
\item 附加条款

``Additional permissions'' are terms that supplement the terms of this
License by making exceptions from one or more of its conditions.
Additional permissions that are applicable to the entire Program shall
be treated as though they were included in this License, to the extent
that they are valid under applicable law.  If additional permissions
apply only to part of the Program, that part may be used separately
under those permissions, but the entire Program remains governed by
this License without regard to the additional permissions.

“附加许可”用于补充本许可,以允许一或者多个例外情况。如果适用于整个程序的附加许可在适用法律的有效,就应该视为本许可协议的一部分。
如果附加许可只适用于程序的某部分,则该部分受此附加许可约束,而其他部分受不适应附加许可之外的条款约束。

When you convey a copy of a covered work, you may at your option
remove any additional permissions from that copy, or from any part of
it.  (Additional permissions may be written to require their own
removal in certain cases when you modify the work.)  You may place
additional permissions on material, added by you to a covered work,
for which you have or can give appropriate copyright permission.

当你传递本程序的副本时,你可以选择性从副本中删除任何附加许可。(在某些情况下,当你修改作品时,附加许可可能已经写明要求你删除该条款。)
对于你传递的作品,如果你拥有或者可以适当地授权,你也可以在作品的材料中添加附加许可。

Notwithstanding any other provision of this License, for material you
add to a covered work, you may (if authorized by the copyright holders of
that material) supplement the terms of this License with terms:
尽管本许可还有的其他条款,对于你添加到受保护作品中的材料,你可以对本许可(如果你获得该材料版权持有人的授权)添加如下补充条款:
  \begin{enumerate}
  \item Disclaiming warranty or limiting liability differently from the
  terms of sections 15 and 16 of this License; or
  \item 以第15条、第16条之外的方式,拒绝提供品质担保或缩小责任。或者

  \item Requiring preservation of specified reasonable legal notices or
  author attributions in that material or in the Appropriate Legal
  Notices displayed by works containing it; or

  \item 要求在此材料中或在法律声明适当中保留特定合理法律声明或作者信息。或者
  
  \item Prohibiting misrepresentation of the origin of that material, or
  requiring that modified versions of such material be marked in
  reasonable ways as different from the original version; or

  \item 禁止对该原始材料不当描述,或要求用不同与原始版本的方式对该材料修改版本合理标示。或者

  \item Limiting the use for publicity purposes of names of licensors or
  authors of the material; or

  \item 限制公开使用授权人或者该材料作者姓名。或者

  \item Declining to grant rights under trademark law for use of some
  trade names, trademarks, or service marks; or

  \item 拒绝使用在商标法下使用商号、商标及服务标识。

  \item Requiring indemnification of licensors and authors of that
  material by anyone who conveys the material (or modified versions of
  it) with contractual assumptions of liability to the recipient, for
  any liability that these contractual assumptions directly impose on
  those licensors and authors.
  \item 任何传递该材料(或其修改版)者,如果对接收者提供契约性责任许诺,需要为授权人或者该材料作者承担赔偿责任,因为任何契约假设责任都造成授权人或者作者承担。

  \end{enumerate}

All other non-permissive additional terms are considered ``further
restrictions'' within the meaning of section 10.  If the Program as you
received it, or any part of it, contains a notice stating that it is
governed by this License along with a term that is a further
restriction, you may remove that term.  If a license document contains
a further restriction but permits relicensing or conveying under this
License, you may add to a covered work material governed by the terms
of that license document, provided that the further restriction does
not survive such relicensing or conveying.

此外的非许可性附加条款都被视作第10条所说的“进一步的限制”。如果你接收到的程序或程序的任何部分,包含受本许可约束的声明,
却补充了这种进一步的限制条款,你可以删除它们。如果某许可文件包含进一步的限制条款,但允许通过该协议再授权或传递,
你可以添加受该许可文件保护的材料,同时提供其他的再许可或者传递的进一步限制条款。

If you add terms to a covered work in accord with this section, you
must place, in the relevant source files, a statement of the
additional terms that apply to those files, or a notice indicating
where to find the applicable terms.

如果你根据本条规定向受保护作品添加了新的条款,你必须在相关的源文件中加入附加条款的对应的声明,或者指明在哪里可以找到适用的条款。

Additional terms, permissive or non-permissive, may be stated in the
form of a separately written license, or stated as exceptions;
the above requirements apply either way.

附加条款,不管是许可性的还是非许可性的,可以以独立的书面协议出现,也可以声明为例外情况,两种做法都可以实现上述要求。

\item Termination.

You may not propagate or modify a covered work except as expressly
provided under this License.  Any attempt otherwise to propagate or
modify it is void, and will automatically terminate your rights under
this License (including any patent licenses granted under the third
paragraph of section 11).

However, if you cease all violation of this License, then your
license from a particular copyright holder is reinstated (a)
provisionally, unless and until the copyright holder explicitly and
finally terminates your license, and (b) permanently, if the copyright
holder fails to notify you of the violation by some reasonable means
prior to 60 days after the cessation.

Moreover, your license from a particular copyright holder is
reinstated permanently if the copyright holder notifies you of the
violation by some reasonable means, this is the first time you have
received notice of violation of this License (for any work) from that
copyright holder, and you cure the violation prior to 30 days after
your receipt of the notice.

Termination of your rights under this section does not terminate the
licenses of parties who have received copies or rights from you under
this License.  If your rights have been terminated and not permanently
reinstated, you do not qualify to receive new licenses for the same
material under section 10.

\item Acceptance Not Required for Having Copies.

You are not required to accept this License in order to receive or
run a copy of the Program.  Ancillary propagation of a covered work
occurring solely as a consequence of using peer-to-peer transmission
to receive a copy likewise does not require acceptance.  However,
nothing other than this License grants you permission to propagate or
modify any covered work.  These actions infringe copyright if you do
not accept this License.  Therefore, by modifying or propagating a
covered work, you indicate your acceptance of this License to do so.

\item Automatic Licensing of Downstream Recipients.

Each time you convey a covered work, the recipient automatically
receives a license from the original licensors, to run, modify and
propagate that work, subject to this License.  You are not responsible
for enforcing compliance by third parties with this License.

An ``entity transaction'' is a transaction transferring control of an
organization, or substantially all assets of one, or subdividing an
organization, or merging organizations.  If propagation of a covered
work results from an entity transaction, each party to that
transaction who receives a copy of the work also receives whatever
licenses to the work the party's predecessor in interest had or could
give under the previous paragraph, plus a right to possession of the
Corresponding Source of the work from the predecessor in interest, if
the predecessor has it or can get it with reasonable efforts.

You may not impose any further restrictions on the exercise of the
rights granted or affirmed under this License.  For example, you may
not impose a license fee, royalty, or other charge for exercise of
rights granted under this License, and you may not initiate litigation
(including a cross-claim or counterclaim in a lawsuit) alleging that
any patent claim is infringed by making, using, selling, offering for
sale, or importing the Program or any portion of it.

\item Patents.

A ``contributor'' is a copyright holder who authorizes use under this
License of the Program or a work on which the Program is based.  The
work thus licensed is called the contributor's ``contributor version''.

A contributor's ``essential patent claims'' are all patent claims
owned or controlled by the contributor, whether already acquired or
hereafter acquired, that would be infringed by some manner, permitted
by this License, of making, using, or selling its contributor version,
but do not include claims that would be infringed only as a
consequence of further modification of the contributor version.  For
purposes of this definition, ``control'' includes the right to grant
patent sublicenses in a manner consistent with the requirements of
this License.

Each contributor grants you a non-exclusive, worldwide, royalty-free
patent license under the contributor's essential patent claims, to
make, use, sell, offer for sale, import and otherwise run, modify and
propagate the contents of its contributor version.

In the following three paragraphs, a ``patent license'' is any express
agreement or commitment, however denominated, not to enforce a patent
(such as an express permission to practice a patent or covenant not to
sue for patent infringement).  To ``grant'' such a patent license to a
party means to make such an agreement or commitment not to enforce a
patent against the party.

If you convey a covered work, knowingly relying on a patent license,
and the Corresponding Source of the work is not available for anyone
to copy, free of charge and under the terms of this License, through a
publicly available network server or other readily accessible means,
then you must either (1) cause the Corresponding Source to be so
available, or (2) arrange to deprive yourself of the benefit of the
patent license for this particular work, or (3) arrange, in a manner
consistent with the requirements of this License, to extend the patent
license to downstream recipients.  ``Knowingly relying'' means you have
actual knowledge that, but for the patent license, your conveying the
covered work in a country, or your recipient's use of the covered work
in a country, would infringe one or more identifiable patents in that
country that you have reason to believe are valid.

If, pursuant to or in connection with a single transaction or
arrangement, you convey, or propagate by procuring conveyance of, a
covered work, and grant a patent license to some of the parties
receiving the covered work authorizing them to use, propagate, modify
or convey a specific copy of the covered work, then the patent license
you grant is automatically extended to all recipients of the covered
work and works based on it.

A patent license is ``discriminatory'' if it does not include within
the scope of its coverage, prohibits the exercise of, or is
conditioned on the non-exercise of one or more of the rights that are
specifically granted under this License.  You may not convey a covered
work if you are a party to an arrangement with a third party that is
in the business of distributing software, under which you make payment
to the third party based on the extent of your activity of conveying
the work, and under which the third party grants, to any of the
parties who would receive the covered work from you, a discriminatory
patent license (a) in connection with copies of the covered work
conveyed by you (or copies made from those copies), or (b) primarily
for and in connection with specific products or compilations that
contain the covered work, unless you entered into that arrangement,
or that patent license was granted, prior to 28 March 2007.

Nothing in this License shall be construed as excluding or limiting
any implied license or other defenses to infringement that may
otherwise be available to you under applicable patent law.

\item No Surrender of Others' Freedom.

If conditions are imposed on you (whether by court order, agreement or
otherwise) that contradict the conditions of this License, they do not
excuse you from the conditions of this License.  If you cannot convey a
covered work so as to satisfy simultaneously your obligations under this
License and any other pertinent obligations, then as a consequence you may
not convey it at all.  For example, if you agree to terms that obligate you
to collect a royalty for further conveying from those to whom you convey
the Program, the only way you could satisfy both those terms and this
License would be to refrain entirely from conveying the Program.

\item Use with the GNU Affero General Public License.

Notwithstanding any other provision of this License, you have
permission to link or combine any covered work with a work licensed
under version 3 of the GNU Affero General Public License into a single
combined work, and to convey the resulting work.  The terms of this
License will continue to apply to the part which is the covered work,
but the special requirements of the GNU Affero General Public License,
section 13, concerning interaction through a network will apply to the
combination as such.

\item Revised Versions of this License.

The Free Software Foundation may publish revised and/or new versions of
the GNU General Public License from time to time.  Such new versions will
be similar in spirit to the present version, but may differ in detail to
address new problems or concerns.

Each version is given a distinguishing version number.  If the
Program specifies that a certain numbered version of the GNU General
Public License ``or any later version'' applies to it, you have the
option of following the terms and conditions either of that numbered
version or of any later version published by the Free Software
Foundation.  If the Program does not specify a version number of the
GNU General Public License, you may choose any version ever published
by the Free Software Foundation.

If the Program specifies that a proxy can decide which future
versions of the GNU General Public License can be used, that proxy's
public statement of acceptance of a version permanently authorizes you
to choose that version for the Program.

Later license versions may give you additional or different
permissions.  However, no additional obligations are imposed on any
author or copyright holder as a result of your choosing to follow a
later version.

\item Disclaimer of Warranty.

\begin{sloppypar}
 THERE IS NO WARRANTY FOR THE PROGRAM, TO THE EXTENT PERMITTED BY
 APPLICABLE LAW.  EXCEPT WHEN OTHERWISE STATED IN WRITING THE
 COPYRIGHT HOLDERS AND/OR OTHER PARTIES PROVIDE THE PROGRAM ``AS IS''
 WITHOUT WARRANTY OF ANY KIND, EITHER EXPRESSED OR IMPLIED,
 INCLUDING, BUT NOT LIMITED TO, THE IMPLIED WARRANTIES OF
 MERCHANTABILITY AND FITNESS FOR A PARTICULAR PURPOSE.  THE ENTIRE
 RISK AS TO THE QUALITY AND PERFORMANCE OF THE PROGRAM IS WITH YOU.
 SHOULD THE PROGRAM PROVE DEFECTIVE, YOU ASSUME THE COST OF ALL
 NECESSARY SERVICING, REPAIR OR CORRECTION.
\end{sloppypar}

\item Limitation of Liability.

 IN NO EVENT UNLESS REQUIRED BY APPLICABLE LAW OR AGREED TO IN
 WRITING WILL ANY COPYRIGHT HOLDER, OR ANY OTHER PARTY WHO MODIFIES
 AND/OR CONVEYS THE PROGRAM AS PERMITTED ABOVE, BE LIABLE TO YOU FOR
 DAMAGES, INCLUDING ANY GENERAL, SPECIAL, INCIDENTAL OR CONSEQUENTIAL
 DAMAGES ARISING OUT OF THE USE OR INABILITY TO USE THE PROGRAM
 (INCLUDING BUT NOT LIMITED TO LOSS OF DATA OR DATA BEING RENDERED
 INACCURATE OR LOSSES SUSTAINED BY YOU OR THIRD PARTIES OR A FAILURE
 OF THE PROGRAM TO OPERATE WITH ANY OTHER PROGRAMS), EVEN IF SUCH
 HOLDER OR OTHER PARTY HAS BEEN ADVISED OF THE POSSIBILITY OF SUCH
 DAMAGES.

\item Interpretation of Sections 15 and 16.

If the disclaimer of warranty and limitation of liability provided
above cannot be given local legal effect according to their terms,
reviewing courts shall apply local law that most closely approximates
an absolute waiver of all civil liability in connection with the
Program, unless a warranty or assumption of liability accompanies a
copy of the Program in return for a fee.

\begin{center}
{\Large\sc End of Terms and Conditions}

\bigskip
How to Apply These Terms to Your New Programs
\end{center}

If you develop a new program, and you want it to be of the greatest
possible use to the public, the best way to achieve this is to make it
free software which everyone can redistribute and change under these terms.

To do so, attach the following notices to the program.  It is safest
to attach them to the start of each source file to most effectively
state the exclusion of warranty; and each file should have at least
the ``copyright'' line and a pointer to where the full notice is found.

{\footnotesize
\begin{verbatim}
<one line to give the program's name and a brief idea of what it does.>

Copyright (C) <textyear>  <name of author>

This program is free software: you can redistribute it and/or modify
it under the terms of the GNU General Public License as published by
the Free Software Foundation, either version 3 of the License, or
(at your option) any later version.

This program is distributed in the hope that it will be useful,
but WITHOUT ANY WARRANTY; without even the implied warranty of
MERCHANTABILITY or FITNESS FOR A PARTICULAR PURPOSE.  See the
GNU General Public License for more details.

You should have received a copy of the GNU General Public License
along with this program.  If not, see <https://www.gnu.org/licenses/>.
\end{verbatim}
}

Also add information on how to contact you by electronic and paper mail.

If the program does terminal interaction, make it output a short
notice like this when it starts in an interactive mode:

{\footnotesize
\begin{verbatim}
<program>  Copyright (C) <year>  <name of author>

This program comes with ABSOLUTELY NO WARRANTY; for details type `show w'.
This is free software, and you are welcome to redistribute it
under certain conditions; type `show c' for details.
\end{verbatim}
}

The hypothetical commands {\tt show w} and {\tt show c} should show
the appropriate
parts of the General Public License.  Of course, your program's commands
might be different; for a GUI interface, you would use an ``about box''.

You should also get your employer (if you work as a programmer) or
school, if any, to sign a ``copyright disclaimer'' for the program, if
necessary.  For more information on this, and how to apply and follow
the GNU GPL, see \texttt{https://www.gnu.org/licenses/}.

The GNU General Public License does not permit incorporating your
program into proprietary programs.  If your program is a subroutine
library, you may consider it more useful to permit linking proprietary
applications with the library.  If this is what you want to do, use
the GNU Lesser General Public License instead of this License.  But
first, please read \texttt{https://www.gnu.org/licenses/why-not-lgpl.html}.

\end{enumerate}

\end{document}
